\documentclass{article}
\usepackage{longtable}
\usepackage{array}

\begin{document}

\begin{longtable}{|>{\raggedright\arraybackslash}p{3cm}|p{5cm}|p{5cm}|p{3cm}|}
    \hline
    \textbf{Name} & \textbf{Description} & \textbf{Tasks} & \textbf{Metrics} \\
    \hline
    \endfirsthead
    \hline
    \textbf{Name} & \textbf{Description} & \textbf{Tasks} & \textbf{Metrics} \\
    \hline
    \endhead
    \hline
    \endfoot
    \hline
    BaisBench (Biological AI Scientist Benchmark) & Designed to assess AI scientists' ability to generate biological discoveries through data analysis and reasoning with external knowledge in omics data-driven research. & Cell type annotation on single-cell datasets, Scientific discovery through multiple-choice questions derived from biological insights of recent single-cell studies. & Accuracy \\
    \hline
    MOLGEN & A benchmark for molecular generation tasks in chemistry and drug discovery. & Generating novel molecules with desired properties. & Diversity, Validity, Novelty, Property scores \\
    \hline
    Open Graph Benchmark (OGB) - Biology & Graph machine learning benchmarks for biological networks (e.g., protein-protein interaction networks, drug-target interactions). & Node classification, Link prediction, Graph classification on biological graphs. & Accuracy, AUC \\
    \hline
    Materials Project & A vast database of material properties, often used as a benchmark for AI models predicting new materials. & Predicting material properties (e.g., bandgaps, formation energies, stability) for inorganic compounds. & Mean Absolute Error (MAE), R-squared \\
    \hline
    OCP (Open Catalyst Project) & Benchmarks for discovering new catalysts using AI, focusing on predicting adsorption energies and forces. & Predicting catalyst properties and reaction outcomes. & MAE on energies and forces \\
    \hline
    JARVIS-Leaderboard (Joint Automated Repository for Various Integrated Simulations) & NIST-maintained leaderboards for AI models in materials science, covering various properties and simulations. & Material property prediction (e.g., superconducting transition temperature), Image classification in STEM, Force field prediction. & Task-specific metrics (e.g., MAE, accuracy) \\
    \hline
    Quantum Computing Benchmarks (e.g., QML Benchmarks) & Evaluates AI models for tasks in quantum computing, such as quantum state preparation, quantum control, and error correction. & Optimizing quantum circuits, Classifying quantum states. & Fidelity, Success probability \\
    \hline
    Fluid Dynamics Benchmarks (e.g., based on CFD data) & Benchmarks for AI models in computational fluid dynamics (CFD), such as predicting flow patterns or turbulence. & Solving Navier-Stokes equations, Predicting aerodynamic forces. & Error metrics (e.g., L2 error) \\
    \hline
    SatImgNet & Benchmark for analyzing satellite imagery, relevant for climate science, disaster monitoring, and urban planning. & Object detection, Semantic segmentation, Change detection in satellite images. & mAP, IoU, Accuracy \\
    \hline
    Climate Model Benchmarks (e.g., forecasting) & Evaluating AI models for climate forecasting, predicting weather patterns, and understanding climate change. & Predicting temperature, Precipitation, Extreme weather events. & RMSE, Bias \\
    \hline
    BIG-Bench (Beyond the Imitation Game Benchmark) & A very large and diverse benchmark that includes many tasks requiring scientific reasoning and knowledge, often pushing the limits of language models. & Problem-solving, Knowledge recall, Common-sense reasoning across a wide array of topics, including scientific ones. & Accuracy, Various task-specific metrics \\
    \hline
    CommonSenseQA & Tests common sense reasoning, which is crucial for scientific understanding and problem-solving. & Answering multiple-choice questions that require context utilization and human-like language understanding. & Accuracy \\
    \hline
    Winogrande & Assesses commonsense reasoning by resolving ambiguities in sentences that require an understanding of context. & Disambiguating sentences based on contextual understanding. & AUC \\
    \hline
\end{longtable}

\end{document}
